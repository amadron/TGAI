% select subfiles base file
\documentclass[TGAI_Laborbericht.tex]{subfiles}
\begin{document}


\chapter{Einleitung}
\label{chap:EINL}
\pagestyle{plain}

Dieser Versuch besteht aus zwei Unterversuchen. Für beide wird ein Mikrofon an das Oszilloskop angeschlossen, mit welchem dann unterschiedliche Frequenzen aufgenommen. Im ersten Versuch ist die Geräuschquelle eine Mundharmonika auf welcher wir einen Ton Spielen. Die gemessenen Daten werden dann vom Oszilloskop an den Laborrechner übertragen und mit einem Python Skript eingelesen. Beim zweiten versuch benutzen wir einen Frequenz/Funktionsgenerator welcher sowohl an einen Lautsprecher als auch an das Oszilloskop angeschlossen ist. Es werden mithilfe des Generators, unterschiedliche Frequenzen im Bereich von 100 Hz bis 10 Khz erzeugt und diese Ausgangswerte dann mit den Eingangswerten verglichen. Die Phasenverschiebung und die Amplitude des Ausgangssignals wird dann von Hand notiert und anschließend am PC grafisch dargestelllt. Zum Schluss wird noch ein Bode Diagramm erzeugt.

\end{document}