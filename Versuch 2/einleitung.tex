% select subfiles base file
\documentclass[TGAI_Laborbericht.tex]{subfiles}
\begin{document}


\chapter{Einleitung}
\label{chap:EINL}
\pagestyle{plain}

In diesem Versuch untersuchen wir die Eigenschaften einer Digitalen Kamera, in unserem Fall eine Webcam der Firma Logitech. Wir werden diese Kalibrieren. Dies geschieht nach dem Verfahren, wie auch bei industriellen Inspektionsanlagen, bei der Fernerkundung durch Satelliten oder in der Astronomie. Zuerst nehmen wir einen Graukeil der verschiedene Graustufen abbildet, parallel zur Kamera auf. Danach wird der Graukeil am PC in seine Graustufen aufgeteilt. Als nächstes nehmen wir ein Dunkelbild auf, damit wir unterschiedliche Nullpunkte, welche manche Pixel haben ausgleichen können. Danach nehmen wir ein Weißbild auf, um die unterschiedliche linearität der Beleuchtungsstärke der einzelnen Pixel zu kompensieren. Zum schluss sollen noch so genannte dead pixel auf dem Weißbild und sogenannte stuck und hot pixel auf dem Schwarzbild gekennzeichnet werden.

\end{document}