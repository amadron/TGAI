% select subfiles base file
\documentclass[TGAI_Laborbericht.tex]{subfiles}
\begin{document}


\chapter{Versuch 1}
\label{chap:VERSUCH_1}


\section{Fragestellung, Messprinzip, Aufbau, Messmittel}
\label{chap:VERSUCH_1_FRAGESTELLUNG}
Zuerst haben wir geschaut, wie gut die Webcam einen Grauwertverlauf aufnimmt und kann da das Muster bekannt ist, die Widergabequalität des Kamerasensors messen. 

\subsection{Messmittel}

Als Messmittel dient eine Webcam der Firma Logitech, welche per USB an den örtlichen Laborrechner angeschlossen ist. Wir steuern die Kamera über das Python modul "OpenCV" an.

\subsection{Messprinzip}

\subsection{Aufbau}


\section{Messwerte}
\label{chap:VERSUCH_1_MESSWERTE}

\section{Auswertung}
\label{chap:VERSUCH_1_AUSWERTUNG}

\section{Interpretation}
\label{chap:VERSUCH_1_INTERPRETATION}

\end{document}