% select subfiles base file
\documentclass[TGAI_Laborbericht.tex]{subfiles}
\begin{document}

\chapter{Versuch 3}
\label{chap:VERSUCH_3}

\section{Fragestellung, Messprinzip, Aufbau, Messmittel}
\label{chap:VERSUCH_3_FRAGESTELLUNG}
Die einzelnen Pixel haben zwar eine hervoragende Linearität mit der Beleuchtung, aber aufgrund von Fertigungstoleranzen ist ihre Sensitivität nicht gleich. Außerdem tritt noch die so genannte Vignettierung auf. Hierbei wird die Helligkeit nicht gleichmäßig auf den Sensor von der Optik übertragen. Dies äußert sich in einer Abdunklung des Bildes zu den Rändern hin. Um diesen Effekt zu kompensieren, nimmt man ein so genanntes Weißbild auf durch welches das normale Bild dividiert wird. Zu beachten ist noch, dass das Weißbild vom Fokus des Kamerasensors abhängt.

\subsection{Aufbau}

Aufbau ist der Selbe wie bisher, bloß dass wir diesmal eine homogene Fläche Brauchen, diese ist in unserem Fall ein weißes Blatt.

\section{Messwerte}
\label{chap:VERSUCH_3_MESSWERTE}

Das aufnehmen der Weißbilder funktioniert ebenfalls wie in Aufgabe 2, mit dem Unterschied, dass wir diesmal anstatt den Sensor zu verdecken, ein weißes Blatt Papier, im gleichen Abstand wie den Graukeil ablichten. Zusätzlich haben wir die Belichtung auf ca. 40\% der Hellsättigung gesetzt.

\section{Auswertung}
\label{chap:VERSUCH_3_AUSWERTUNG}

Nun lesen wir ebenfalls wieder die Bilder ein und bilden den Mittelwert mit dem gleichen Verfahren wie schon in Aufgabe 2. Dadurch erhalten wir ein Weißbild. Das Skript zur Korrektur des Graukeils wird nun so erweitert, dass das Weißbild ebenfalls eingelesen wird. Dieses wird nun in Double umgewandelt und anschließend normiert. Die Normierung erfolgt über die Berechnung des Mittelwertes durch den das Bild dividiert wird.

\section{Interpretation}
\label{chap:VERSUCH_3_INTERPRETATION}

Nun haben wir ein Weißbild erstellt auf dem man dead pixel erkennen kann, welche wir in unserem Fall nicht haben. Das python skript  zur Korrektur des Graukeils wurde so erweitert, dass nicht nur stuck und hot pixel sondern auch dead pixel eliminiert werden.

\end{document}