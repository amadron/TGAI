% select subfiles base file
\documentclass[TGAI_Laborbericht.tex]{subfiles}
\begin{document}


\chapter{Versuch 2}
\label{chap:VERSUCH_2}


\section{Fragestellung, Messprinzip, Aufbau, Messmittel}
\label{chap:VERSUCH_2_FRAGESTELLUNG}
\subsection{Fragestellung}

Im zweiten Teil des Versuchs, erstellen wir ein Dunkelbild, dies ist ein Bild bei kompletter Verdunklung des Sensors. Dieses brauchen wir, weil bei nicht jeder Pixel den Grauwert 0 hat wenn der Sensor verdeckt wird. Das liegt am thermischen Rauschen der Ausleseelektronik und am Dunkelstrom. So müssen wir ein Dunkelbild erstellen um diesen Effekt aus normalen Bildern zu eliminieren.


\subsection{Aufbau}

Am Aufbau selbst hat sich nicht viel geändert. Nun brauchen wir noch zusätzlich ein Objekt, um den Sensor zu verdecken.

\subsection{Messmittel}

Als Messmitel dient wieder die Webcam aus Versuch 1.

\section{Messwerte}
\label{chap:VERSUCH_2_MESSWERTE}

\section{Auswertung}
\label{chap:VERSUCH_2_AUSWERTUNG}

\section{Interpretation}
\label{chap:VERSUCH_2_INTERPRETATION}

\end{document}