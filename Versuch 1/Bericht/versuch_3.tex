% select subfiles base file
\documentclass[TGAI_Laborbericht.tex]{subfiles}
\begin{document}

\chapter{Versuch 3}
\label{chap:VERSUCH_3}

\section{Fragestellung, Messprinzip, Aufbau, Messmittel}
\label{chap:VERSUCH_3_FRAGESTELLUNG}
\subsection{Fragestellung}
Im letzten Teil des Versuches, soll nun die Länge und Breite eines DIN A 4 Blattes anhand der Ergebnisse aus Versuch 2 ermittelt werden. Dazu soll noch die Fläche des Blattes berechnet werden. Außerdem sollen wir noch den Messfehler mithilfe der Fehlerfortpflanzung ermitteln.
\section{Messwerte}
\label{chap:VERSUCH_3_MESSWERTE}
Wir haben mithilfe des Sensors, die Breite und die Länge eines DIN A4 Blattes gemessen und diese an den Computer per USB übertragen. Die Entfernung in cm wird mithilfe des ermittelten Zusammenhangs aus Aufgabe2 errechnet.
\section{Auswertung}
\label{chap:VERSUCH_3_AUSWERTUNG}
Nachdem wir die Länge und Breite des DIN A4 Blattes mithilfe des Sensors gemessen haben, müssen wir zuerst die Werte in cm und anschließend Messfehler berechnen. Die Länge und Breite erhalten wir, indem wir die Umkehrfunktion aus Aufgabe 2 nach den cm werten umstellen und die gemessenen Werte einsetzen.
So erhielten wir für die Länge 30,26cm und für die Breite 21,19cm.
Nun berechnen wir noch den Messfehler. Diesen berechnen wir wie in der Vorlesung besprochen mithilfe der Formel

\begin{equation}
X = \overline{X} \pm t * s 
\end{equation}

Man berechnet jeweils den Wert für den Vertrauensbereich von 68\% sowie 95\%. Hierzu Multipliziert man die Standardabweichung mit dem Korrekturfaktor, welchen wir aus den Vorlesungsunterlagen entnehmen. Zum Schluss müssen wir noch die Fläche ausrechnen. Hierzu multipliziert man die Breite mit der Länge und in unserem Fall erhalten wir eine Fläche von 641,34 cm². Nun berechnen wir hier auch wieder den Messfehler mithilfe der Gaußschen Fehlerfortpflanzungsgesetz und erhalten hier für 68\% $\pm$ 29,36 und für 95\% $\pm$ 57,54 cm².
\section{Interpretation}
\label{chap:VERSUCH_3_INTERPRETATION}
Durch die Bildung der Kennlinie in Versuch 2, ist es uns nun möglich Abstände zu messen. Aufgrund der Schwankung der Messergebnisse bekommen wir eine recht hohe Standardabweichung, die sich vor allem in der Fehlerfortpflanzung bemerkbar macht.
Dies zeigt sich auch, wenn man die berechnete Fläche (641,40cm²) mit der eigentlichen Fläche eines DIN A4 Blattes vergleicht (623,7cm²). Hier kann man sehr gut erkennen, dass sich die eigentliche Fläche eines DIN A4 Blattes zwar in dem zu berechnenten Bereich liegt jedoch stark von seinem Mittelwert abweicht.
\end{document}