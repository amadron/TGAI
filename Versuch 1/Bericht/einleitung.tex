% select subfiles base file
\documentclass[TGAI_Laborbericht.tex]{subfiles}
\begin{document}


\chapter{Einleitung}
\label{chap:EINL}
\pagestyle{plain}
Ziel des Versuches war es, den in der Vorlesung vermittelten Stoff anhand eines praktischen
Beispieles umzusetzen. Es war gefordert, mithilfe eines Abstandssensors innerhalb eines
Messbereiches von 10-70cm den Abstand zu einem Objekt zu messen sowie die
Seitenlängen eines Blattes und die Eingangswerte mit den Ausgangswerten zu vergleichen.
Anschließend wurden die Werte am PC eingelesen und in Python als Plot dargestellt. Die
ermittelten Werten wurden per logaritmierung und linearisierung in die Form einer
Ausgleichsgeraden gebracht. Durch exponensierung erhält man die Ausgleichsfunktion.
Anhand dieser Ausgleichsfunktion bestimmen wir die Länge eines DinA4 Blattes inklusive
der Standartabweichung. Als letzte Disziplin bestimmen wir mit der Länge und der Breite
des DinA4 Blatts, dessen Fläche. Wobei hier die Fehlerfortpflanzung mit einberechnet
wurde.


\end{document}