% select subfiles base file
\documentclass[TGAI_Laborbericht.tex]{subfiles}
\begin{document}


\chapter{Einleitung}
\label{chap:EINL}
\pagestyle{plain}
In diesem Versuch wird ein einfacher Spracherkenner aufgebaut, der z.B. zur Steuerung eines
Staplers in einem Hochregallager dienen könnte. Es reichen hierzu die Erkennung der vier ein-
fachen Befehle ”Hoch”, ”Tief”, ”Links” und ”Rechts”. Der Aufbau des Spracherkenners folgt
dem in der Vorlesung beschriebenen Prinzip des Prototyp-Klassifikators. Die zugehörigen
Spektren werden mit der Windowing-Methode berechnet.

\end{document}