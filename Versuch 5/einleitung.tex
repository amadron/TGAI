% select subfiles base file
\documentclass[TGAI_Laborbericht.tex]{subfiles}
\begin{document}


\chapter{Einleitung}
\label{chap:EINL}
\pagestyle{plain}

In diesem Versuch arbeiten wir mit einer Multifunktionsbox, die sowohl als Analog-Digital Wandler als auch Digital-Analog Wandler betrieben werden kann. Zuerst betreiben wir das Gerät als Analog-Digital Wandler und berechnen den Quantisierungsfehler als auch die Standardabweichung. Im nächsten Teil berechnen wir nochmals den Quantisierungsfehler, aber diesmal vom Digital-Analog Wandler, und vergleichen eine ausgegebene Spannung mit der eingestellten Spannung. Zum Schluss geben wir mithilfe des Digital-Analog Wandlers eine Sinusspannung aus. 

\end{document}